%----------------------------------------------------------------------------------------
%	PACKAGES AND OTHER DOCUMENT CONFIGURATIONS
%----------------------------------------------------------------------------------------

\documentclass{article}

\usepackage{lastpage} % Required to determine the last page for the footer
\usepackage{extramarks} % Required for headers and footers
\usepackage{graphicx} % Required to insert images
\usepackage{lipsum} % Used for inserting dummy 'Lorem ipsum' text into the template
\usepackage{hyperref}

\linespread{1.1} % Line spacing
\usepackage[total={6in, 9in}]{geometry}

\setlength\parindent{0pt} % Removes all indentation from paragraphs



\begin{document}
\title{Advanced Computer Graphics\\ Coursework 1}
\author{Terence Tse, Zhou Yu \\ Team JT}
\maketitle
\newpage

\section{Assembling a HDR image}

\subsection{Our center-weighting function}
Our center-weighting funcion, \texttt{w}, can be defined as below:
\begin{verbatim}
Fucking sick m8
\end{verbatim}
The reason we chose our function to be this way was

\subsection{Image after loading and weighting}
\begin{center}
	\begin{figure}[H]
		\begin{center}
			\includegraphics[scale=0.33]{Memorial/updatedExposureGama.ppm}
			\caption{HDR image}
		\end{center}
	\end{figure}
\end{center}

\subsubsection{ Dynamic Range of scene} 
%% ratio of brightest pixels to the dimmest

%\subsection{Tone Mapping}
%\begin{center}
%	\begin{figure}[H]
%		\begin{center}
%			\includegraphics[scale=0.33]{}
%			\caption{HDR image after Tone mapping, f.ppm}
%		\end{center}
%	\end{figure}
%\end{center}

\subsubsection{Choosing Parameters}
After investigation into the appearances of the resulting pictures 
by changing the number of stops to apply to the image, we chose
the value, \texttt{n}, to be XXXXXXXXXXXXXXXX. 

In addition, we checked out the appearance of the image under 
different values of gamma correction. We tested the values suggested
in the specification as well as XXXXXXXXXXXXXXX. We decided to leave
our gamma as XXXXXXXXXXXXXX.

%\section{Implmenting simple Image Based Lighting}
%\subsection{Creating the surface normal picture}
%\begin{center}
%	\begin{figure}[H]
%		\begin{center}
%			\includegraphics[scale=0.33]{}
%			\caption{Surface Normals (XYZ \Rightarrow RGB), something.pfm}
%		\end{center}
%	\end{figure}
%\end{center}

%\subsubsection{Calculating reflectance vector picture}
%\begin{center}
%	\begin{figure}[H]
%		\begin{center}
%			\includegraphics[scale=0.33]{}
%			\caption{Reflectance Vectors (XYZ \Rightarrow RGB), something.pfm}
%		\end{center}
%	\end{figure}
%\end{center}
%
%\subsubsection{Converting to ppm}
%\begin{center}
%	\begin{figure}[H]
%		\begin{center}
%			\includegraphics[scale=0.33]{}
%			\caption{Reflectance Vectors as PPM, something.ppm}
%		\end{center}
%	\end{figure}
%\end{center}
%
%\subsubsection{Applying the lighting environment of Grace Cathedral}
%\begin{center}
%	\begin{figure}[H]
%		\begin{center}
%			\includegraphics[scale=0.33]{}
%			\caption{Grace Cathedral lighting has been applied, GC.pfm}
%		\end{center}
%	\end{figure}
%\end{center}
%
%\begin{center}
%	\begin{figure}[H]
%		\begin{center}
%			\includegraphics[scale=0.33]{}
%			\caption{Grace Cathedral lighting has been applied, GeorgiannaChan.pfm}
%		\end{center}
%	\end{figure}
%\end{center}

\end{document}
